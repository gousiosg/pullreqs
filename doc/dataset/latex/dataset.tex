\documentclass{sig-alternate}
\usepackage{url}
\usepackage{graphicx}
\usepackage{subfigure}
\usepackage{hyperref}
\usepackage{url}
\usepackage{times}
\usepackage{balance}
\usepackage{xspace}
%\usepackage{xfrac}

\begin{document}

\newcommand{\todo}[1]{\textbf{TODO}\footnote{\textbf{TODO:} #1}}

\newcommand{\ghtorrent}{ \textsc{ght}orrent\xspace}
\newcommand{\api}{\textsc{api}\xspace}
\newcommand{\pullreq}{ \textsf{pullreq}\xspace}

\title{A dataset for pull-based development research}
\numberofauthors{3}

\author{
\alignauthor
Georgios Gousios\\
       \affaddr{Delft University of Technology}\\
       \affaddr{Delft, The Netherlands}\\
       \email{G.Gousios@tudelft.nl}
}

\maketitle

\begin{abstract}


\end{abstract}

\category{D.2.7}{Software Engineering}{Distribution, Maintenance, and Enhancement}[Version control]
\category{D.2.9}{Software Engineering}{Management}[Programming teams]

\terms{Management}

\keywords{pull-based development, pull request, distributed software development,
empirical software engineering}

\section{Introduction}
\label{sec:intro}

\section{Data collection process}
\label{sec:expdata}

To create the dataset, we followed an iterative process; first, we selected all
projects in the \ghtorrent dataset that by the end of 2013 had more than 100
pull requests. Then, the full history (including pull requests, issues and
commits) of the included projects was downloaded and features were extracted by
querying the {\sc ght}orrent databases and analyzing each project's Git
repository. The initial selection resulted in 1047 projects. The following
criteria were then applied to exclude projects from the initial selection:

\begin{itemize}

  \item After processing, the number of remaining pull requests should be more
    that 80. The typical reason for some pull requests missing from the output
    is  \todo{Find a reason} . 

  \item Projects should include tests. We could only use projects that include
    tests which we could measure reliably. For that, we exploited the
    convention-based project layout in the Ruby (Gem), Python, Java and Scala
    (both Maven) language ecosystems, so our project selection was limited to
    those languages. After processing, we filtered out 97 projects for which
    our heuristics could not identify any tests.
    
  \item Projects should have at least one commit coming from developer outside
    the project's main team, to ensure that the project is open to external
    contributions and that pull requests are not just used by developers within
    the project.

  \item The ratio of merged pull requests should be within reasonable limits
    from the mean merge ratio across all Github projects. In the \ghtorrent
    dataset, the mean merge ratio is 72\%. The \pullreq dataset uses heuristics
    to indetify merges done outside Github (see Section~\ref{sec:mergedec}), so
    we generally expect a higher average merge ratio. We set an arbitrary limit
    of 50\% being merged below which projects are excluded by the dataset. 47
    projects where filtered out.

\end{itemize}

The final dataset consisted of 976 projects (328 Python, 324 Java, 288 Ruby, 37
Scala) and 351,286 pull requests (59,970; 55,468; 43,870 and 7,576 for Python,
Ruby, Java and Scala projects respectively). Both distributions are
representative of the contemporary popularity of each respective programming
language on both Github and other sites.

Figure~\ref{fig:wordcloud} presents the project names in relative size to the
pull requests included per project in the dataset. In the following sections, we descibe important aspects of the dataset generation process.

\begin{figure}
  \begin{center}
    \includegraphics[scale=0.5]{wordcloud.pdf}
  \end{center}
  \caption{Projects in the \pullreq dataset. Text size is relative to the number of
  included pull requests.}
  \label{fig:wordcloud}
\end{figure}

\subsection{Merge detection}
\label{sec:mergedec}
To identify merged pull requests that are merged outside Github, we resorted to
the following heuristics, listed here in order of application:

\begin{description}

  \item[$H_1$] At least one of the commits associated with the pull request appears in the target project's master branch.

  \item[$H_2$] A commit closes the pull request (using the \texttt{fixes:}
    convention advocated by Github) and that commit appears in the project's
    master branch.  This means that the pull request commits were squashed onto
    one commit and this commit was merged.

  \item[$H_3$] One of the last 3 (in order of appearance) discussion comments contain a
    commit unique identifier, this commit appears in the project's master branch
    and the corresponding comment can be matched by the following regular
    expression:

    \begin{small}
    \texttt{(?:merg|appl|pull|push|integrat)(?:ing|i?ed)}
    \end{small}

  \item[$H_4$] The latest comment prior to closing the pull request matches the
    regular expression above.

\end{description}

If none of the above heuristics identifies a merge, the pull request is
identied as unmerged. In the dataset, 58\% of the pull requests are 
merged using Github's facilities while 18\% are identified as unmerged.
The remaining 24\% is identied as merged using the heuristics described
above. The contribution of each heuristic 

\subsection{Test case detection}

To count testing related data, the script exploits the fact that Java, Ruby and Python projects are organized using the Maven, Gem and Pythonic project conventions respectively. Test cases are recognized as follows:

\begin{description}

  \item[Java]  Files in directories under a /test/ branch of the file tree are considered test files. JUnit 4 test cases are recognized using the @Test tag. For JUnit3, methods starting with test are considered as test methods. Asserts are counted by "grepping" through the source code lines for assert* statements.

  \item[Python]

  \item[Ruby] Files under the /test/ and /spec/ directories are considered test files. Test cases are recognized by "grepping" for test* (RUnit), should .* do (Shoulda) and it .* do (RSpec) in the source file lines.


  \item[Scala] Same as Java with the addition of specs2 matchers

\end{description}

\subsection{Counting lines, files and file types}

\subsection{Feature Extraction.} 

\begin{figure}
  \begin{center}
    \includegraphics[scale=0.6]{cross-cor.pdf}
  \end{center}
  \caption{Cross-correlation among all dataset features}
  \label{fig:wordcloud}
\end{figure}


The feature selection was based on prior work in the areas of patch submission
and acceptance~\cite{Nagap05,Bird07a,Weiss08,Baysa12}, code
reviewing~\cite{Rigby13}, bug triaging~\cite{Anvik06, Giger10} and also on
semi-structured interviews of Github developers~\cite{Dabbi12, Pham13, McDon13}.
The selected features are split into three categories:

  \emph{Pull request characteristics.} These features attempt to quantify the
  impact of the pull request on the affected code base. When examining external
  code contributions, the size of the patch is affecting both acceptance and
  acceptance time~\cite{Weiss08}. There are various metrics to determine the
  size of a patch that have been used by researchers: code churn~\cite{Nagap05,
  Ratzi07}, changed files~\cite{Nagap05} and number of commits~\cite{Fluri07}.
  In the particular case of pull requests, developers reported that the presence
  of tests in a pull request increases their confidence to merge
  it~\cite{Pham13}. To investigate this, we split the churn feature into two
  features, namely \texttt{src\_churn} and \texttt{test\_churn}. The
  number of participants has been shown to influence the time to process of code
  reviewing~\cite{Rigby13}. Finally, through our own experience analyzing pull
  requests, we have found that in many cases conflicts are reported explicitly
  in pull request comments while in other cases pull requests include links to
  other related pull requests.

  \emph{Project characteristics.} These features quantify how receptive to pull
  requests the project is. If the project's process is open to external
  contributions, then we expect to see an increased ratio of external
  contributors over team members. The project's size may be a detrimental factor
  to the speed of processing a pull request, as its impact may be more difficult
  to assess. Also, incoming changes tend to cluster over time (the ``yesterday's
  weather'' change pattern~\cite{Girba04}), so it is natural to assume that pull
  requests affecting a part of the system that is under active development will
  be more likely to merge. Testing plays a role in speed of processing;
  according to~\cite{Pham13}, projects struggling with a constant flux of
  contributors use testing, manual or preferably automated, as a safety net to
  handle contributions from unknown developers.

  \emph{Developer.}  Developer-based features quantify the influence that the
  person who created the pull request has on the decision to merge it and the
  time to process it. In particular, the developer who created the patch has
  been shown to influence the patch acceptance decision~\cite{Jeong09}. To
  abstract the results across projects with different developers, we include
  features that quantify the developer's track record~\cite{Dabbi12}, namely the
  number of previous pull requests and their acceptance rate; the former has
  been identified as a strong indicator of pull request quality~\cite{Pham13}.
  Bird et al.~\cite{Bird07}, presented evidence that social reputation has an
  impact on whether a patch will be merged; in our dataset, the number of
  followers on Github can be seen as a proxy for reputation.

All features are calculated at the time a pull request has been closed or
merged, to evaluate the effect of intermediate updates to the pull request as a
result of the ensuing discussion. Features that contain a temporal dimension in
their calculation (e.g., \texttt{team\_size} or
\texttt{commits\_on\_files\_touched}) are calculated over the three-month time
period before the pull request was opened.

% latex table generated in R 2.15.3 by xtable 1.7-1 package
% Tue Jan  7 14:58:43 2014
\begin{table*}[ht]
\centering
\begin{tabular}{rp{15em}ccccc}
  \hline
Feature & Description & quant\_5 & mean & median & quant\_95 & histogram \\ 
  \hline
num\_commits & Number of commits in the pull request & 1.00 & 4.43 & 1.00 & 12.00 & \includegraphics[scale = 0.1, clip = true, trim= 50px 60px 50px 60px]{hist-f128f3cb38588fe5202716588c047381.pdf} \\ 
  src\_churn & Number of lines changed (added + deleted) by the pull request. & 0.00 & 286.91 & 10.00 & 865.00 & \includegraphics[scale = 0.1, clip = true, trim= 50px 60px 50px 60px]{hist-1f006c80a0da61518435a0c55f538326.pdf} \\ 
  test\_churn & Number of test lines changed in the pull request. & 0.00 & 82.59 & 0.00 & 256.00 & \includegraphics[scale = 0.1, clip = true, trim= 50px 60px 50px 60px]{hist-dd78ccaeedd7fc79735a66eb7f9e506b.pdf} \\ 
  files\_changed & Number of files touched by the pull request. & 1.00 & 11.80 & 2.00 & 30.00 & \includegraphics[scale = 0.1, clip = true, trim= 50px 60px 50px 60px]{hist-9b07b060359435635ff2bf4cd34f834a.pdf} \\ 
  num\_comments & The total number of comments (discussion and code review). & 0.00 & 2.59 & 1.00 & 11.00 & \includegraphics[scale = 0.1, clip = true, trim= 50px 60px 50px 60px]{hist-9db5e2b390de0d64d26c14798cb579ef.pdf} \\ 
  num\_participants & Number of participants in the pull request discussion & 0.00 & 1.27 & 1.00 & 4.00 & \includegraphics[scale = 0.1, clip = true, trim= 50px 60px 50px 60px]{hist-7d419bb69f175ea7015a9bdc71172f38.pdf} \\ 
  sloc & Executable lines of code at pull request merge time. & 469.00 & 56609.81 & 18493.50 & 295609.90 & \includegraphics[scale = 0.1, clip = true, trim= 50px 60px 50px 60px]{hist-6b5159d3060b4fdf8493d4c818f79949.pdf} \\ 
  team\_size & Number of active core team members during the last 3 months prior the pull request creation. & 1.00 & 21.11 & 7.00 & 93.00 & \includegraphics[scale = 0.1, clip = true, trim= 50px 60px 50px 60px]{hist-231fb4fabf4a3f0c551f2a97ae080508.pdf} \\ 
  perc\_external\_contribs & The ratio of commits from external members over core team members in the last 3 months prior to pull request creation. & 8.00 & 53.90 & 56.00 & 95.00 & \includegraphics[scale = 0.1, clip = true, trim= 50px 60px 50px 60px]{hist-a222f0a5c377ba129dd6c8f257062591.pdf} \\ 
  commits\_on\_files\_touched & Number of total commits on files touched by the pull request 3 months before the pull request creation time. & 0.00 & 51.49 & 4.00 & 208.00 & \includegraphics[scale = 0.1, clip = true, trim= 50px 60px 50px 60px]{hist-b735900ffcc37e7eda16dcd0c3497e6e.pdf} \\ 
  test\_lines\_per\_kloc & A proxy for the project's test coverage. & 0.00 & 1289.65 & 356.87 & 2093.26 & \includegraphics[scale = 0.1, clip = true, trim= 50px 60px 50px 60px]{hist-67ff3047089ba9ce0528884eab66e80a.pdf} \\ 
  prev\_pullreqs & Number of pull requests submitted by a specific developer, prior to the examined pull request. & 0.00 & 42.79 & 11.00 & 195.00 & \includegraphics[scale = 0.1, clip = true, trim= 50px 60px 50px 60px]{hist-a2f7f60851dfa13cfbe0227d1d233767.pdf} \\ 
  requester\_succ\_rate & The percentage of the developer's pull requests that have been merged up to the creation of the examined pull request. & 0.00 & 0.51 & 0.61 & 1.00 & \includegraphics[scale = 0.1, clip = true, trim= 50px 60px 50px 60px]{hist-9363017165c3ded62457750f1c67c1af.pdf} \\ 
   \hline
\end{tabular}
\caption{Selected features and descriptive statistics. Historgrams are in log scale.} 
\label{tab:features}
\end{table*}


\section{Tools}

The \pullreq dataset is accompanied by an extensive analysis toolkit
written in the R statistics language.

\section{Challenges and limitations}
\label{sec:challenges}

\paragraph*{Merge detection is incomplete}

\paragraph*{Projects are incomplete}

\paragraph*{Test detection is approximate}


\section{Research opportunities}
\label{sec:discussion}

\section{Related Work}
\label{sec:rel}


\section{Conclusion}

%\subsection*{Acknowledgements}

%We would like to thank the anonymous reviewers for their comments.
%This work is partially supported by Marie Curie {\sc ief} 298930 --- {\sc sefunc}.

\bibliographystyle{abbrv}
\balance
%\begin{small}

  \bibliography{dataset}
%\end{small}

\end{document}
