\documentclass{sig-alternate}
\usepackage{url}
\usepackage{graphicx}
\usepackage{subfigure}
\usepackage{hyperref}
\usepackage{url}
\usepackage{times}
\usepackage{balance}
\usepackage{xspace}
%\usepackage{xfrac}

\begin{document}

\newcommand{\todo}[1]{\textbf{TODO}\footnote{\textbf{TODO:} #1}}

\newcommand{\ghtorrent}{ \textsc{ght}orrent\xspace}
\newcommand{\api}{\textsc{api}\xspace}
\newcommand{\pullreqs}{ \textsf{pullreqs}\xspace}

\title{A dataset for pull-based development research}
\numberofauthors{3}

\author{
\alignauthor
Georgios Gousios\\
       \affaddr{Delft University of Technology}\\
       \affaddr{Delft, The Netherlands}\\
       \email{G.Gousios@tudelft.nl}
}

\maketitle

\begin{abstract}

Pull requests form a new method for collaborating on distributed software
development. To study the pull request distributed development model, we
constructed a dataset of almost 800 projects, including some of the largest
users of pull requests on Github. In this paper, we describe how the project
selection was done, we analyze the selected features and present a machine
learning tool set for the R statistics environment. 

\end{abstract}

\category{D.2.7}{Software Engineering}{Distribution, Maintenance, and Enhancement}[Version control]
\category{D.2.9}{Software Engineering}{Management}[Programming teams]

\terms{Management}

\keywords{pull-based development, pull request, distributed software development,
empirical software engineering}

\section{Introduction}
\label{sec:intro}

Pull requests as a distributed development model in general, and as implemented
by Github in particular, form a new method for collaborating on distributed
software development. In the pull-based development model, the project's main
repository is not shared among potential contributors; instead, contributors
\emph{fork} (clone) the repository and make their changes independent of each
other. When a set of changes is ready to be submitted to the main repository,
they create a pull request, which specifies a local branch to be merged with a
branch in the main repository. A member of the project's core team is then
responsible to inspect the changes and pull them to the project's master branch.
If changes are considered unsatisfactory, more changes may be requested; in that
case, contributors need to update their local branches with new commits.
Furthermore, as pull requests only specify branches from which certain commits
can be pulled, there is nothing that forbids their use in the shared 
repository approach (cross-branch pull requests).

To understand what the underlying principles that guide pull-based development
are, we created \pullreqs, a curated dataset of almost 800 projects along with a
set of tools for its analysis. A previous version of the dataset has been used
to quantitatively study the pull request development process~\cite{GPD14}. The
\pullreqs dataset is based on our previous work on \ghtorrent~\cite{Gousi13}, albeit only for
its construction. In this paper, we describe the construction process of the
dataset and outline directions for further research with it.

\section{Data collection process}
\label{sec:expdata}

To create the dataset, we followed an iterative process; first, we selected all
projects in the \ghtorrent dataset which by the end of 2013 had more than 100
pull requests. The initial selection resulted in 1047 projects. For those, the
full history (including pull requests, issues and commits) of the included
projects was downloaded and features were extracted by querying the {\sc
ght}orrent databases and analyzing each project's Git repository. The following
criteria were then applied to exclude projects from the initial selection:

\begin{itemize}

  \item After processing, the number of remaining pull requests should be more
    that 80. The typical reason for some pull requests missing from the output
    is that many projects use a dedicated branch for documentation which
    includes no source code; the data generation script skips such pull
    requests.  87 projects were filtered out

  \item Projects should include tests. We could only use projects that include
    tests which we could measure reliably. For that, we exploited the
    convention-based project layout in the Ruby (Gem), Python, Java and Scala
    (both Maven) language ecosystems, so our project selection was limited to
    those languages. After processing, we filtered out 97 projects for which
    our heuristics could not identify any tests.
    
  \item Projects should have at least one commit coming from developer outside
    the project's main team, to ensure that the project is open to external
    contributions and that pull requests are not just used by developers within
    the project.

  \item The ratio of merged pull requests should be within reasonable limits
    from the mean merge ratio across all Github projects. In the \ghtorrent
    dataset, the mean merge ratio is 72\%. The \pullreqs dataset uses heuristics
    to identify merges done outside Github (see Section~\ref{sec:mergedec}), so
    we generally expect a higher average merge ratio. We set an arbitrary limit
    of 50\% being merged below which projects are excluded by the dataset. 47
    projects where filtered out.

\end{itemize}

The final dataset consisted of 976 projects (328 Python, 324 Java, 288 Ruby, 37
Scala) and 351,286 pull requests (59,970; 55,468; 43,870 and 7,576 for Python,
Ruby, Java and Scala projects respectively). Both distributions are
representative of the contemporary popularity of each respective programming
language on both Github and other sites.

%Figure~\ref{fig:wordcloud} presents the project names in relative size to the
%pull requests included per project in the dataset. In the following sections, we descibe important aspects of the dataset generation process.
%
%\begin{figure}
%  \begin{center}
%    \includegraphics[scale=0.5]{wordcloud.pdf}
%  \end{center}
%  \caption{Projects in the \pullreq dataset. Text size is relative to the number of
%  included pull requests.}
%  \label{fig:wordcloud}
%\end{figure}
%
\subsection{Feature Selection} 

\begin{figure}
  \begin{center}
    \includegraphics[scale=0.6]{cross-cor.pdf}
  \end{center}
  \caption{Cross-correlation (Spearman) among all dataset features. Blue color indicates positive correlation, red color is negative correlation. The darker
  the color (and the more acute the ellipsis slant), the stronger the correlation.}
  \label{fig:features}
\end{figure}

The feature selection was based on prior work in the areas of patch submission
and acceptance~\cite{Baysa12}, code
reviewing~\cite{Rigby13}, bug triaging~\cite{Giger10} and also on
semi-structured interviews of Github developers~\cite{Dabbi12, Pham13}.
The selected features are split into three categories:

  \emph{Pull request characteristics.} These features attempt to quantify the
  impact of the pull request on the affected code base. When examining external
  code contributions, the size of the patch is affecting both acceptance and
  acceptance time. There are various metrics to determine the size of a patch
  that have been used by researchers: code churn, changed files and number of
  commits.
  
  We split coarse-grained metrics in their constituents: for example,
  in the particular case of pull requests, developers reported that
  the presence of tests in a pull request increases their confidence to merge
  it~\cite{Pham13}. To investigate this, we split the churn feature into two
  features, namely \texttt{src\_churn} and \texttt{test\_churn}. The number of
  participants has been shown to influence the time to process of code
  reviewing~\cite{Rigby13}. Therefore, the number of comments in pull requests
  is split between code review and discussion.

  Finally, through our own experience analyzing pull
  requests, we have found that in many cases conflicts are reported explicitly
  in pull request comments while in other cases pull requests include links to
  other related pull requests.

  \emph{Project characteristics.} These features quantify how receptive to pull
  requests the project is. If the project's process is open to external
  contributions, then we expect to see an increased ratio of external
  contributors over core team members. The project's size may be a detrimental
  factor to the speed of processing a pull request, as its impact may be more
  difficult to assess. Also, incoming changes tend to cluster over time (the
  ``yesterday's weather'' change pattern), so it is natural to
  assume that pull requests affecting a part of the system that is under active
  development will be more likely to merge. Testing plays a role in speed of
  processing; according to~\cite{Pham13}, projects struggling with a constant
  flux of contributors use testing, manual or preferably automated, as a safety
  net to handle contributions from unknown developers.

  \emph{Developer.}  Developer-based features quantify the influence that the
  person who created the pull request has on the decision to merge it and the
  time to process it. In particular, the developer who created the patch has
  been shown to influence the patch acceptance decision. To
  abstract the results across projects with different developers, we include
  features that quantify the developer's track record~\cite{Dabbi12}, namely the
  number of previous pull requests and their acceptance rate; the former has
  been identified as a strong indicator of pull request quality~\cite{Pham13}.
  Bird et al. presented evidence that social reputation has an
  impact on whether a patch will be merged; in our dataset, the number of
  followers on Github can be seen as a proxy for reputation.

All features are calculated at the time a pull request has been closed or
merged, to evaluate the effect of intermediate updates to the pull request as a
result of the ensuing discussion. Features that contain a temporal dimension in
their calculation (e.g., \texttt{team\_size} or
\texttt{commits\_on\_files\_touched}) are calculated over the three-month time
period before the pull request was opened. As Figure~\ref{fig:features} shows,
the resulting features are fairly orthogonal, with very few strong correlations
between them. This is an indication that they capture a wide range of pull
request activities with minimal overlap.

% latex table generated in R 2.15.3 by xtable 1.7-1 package
% Tue Jan  7 14:58:43 2014
\begin{table*}[ht]
\centering
\begin{tabular}{rp{15em}ccccc}
  \hline
Feature & Description & quant\_5 & mean & median & quant\_95 & histogram \\ 
  \hline
num\_commits & Number of commits in the pull request & 1.00 & 4.43 & 1.00 & 12.00 & \includegraphics[scale = 0.1, clip = true, trim= 50px 60px 50px 60px]{hist-f128f3cb38588fe5202716588c047381.pdf} \\ 
  src\_churn & Number of lines changed (added + deleted) by the pull request. & 0.00 & 286.91 & 10.00 & 865.00 & \includegraphics[scale = 0.1, clip = true, trim= 50px 60px 50px 60px]{hist-1f006c80a0da61518435a0c55f538326.pdf} \\ 
  test\_churn & Number of test lines changed in the pull request. & 0.00 & 82.59 & 0.00 & 256.00 & \includegraphics[scale = 0.1, clip = true, trim= 50px 60px 50px 60px]{hist-dd78ccaeedd7fc79735a66eb7f9e506b.pdf} \\ 
  files\_changed & Number of files touched by the pull request. & 1.00 & 11.80 & 2.00 & 30.00 & \includegraphics[scale = 0.1, clip = true, trim= 50px 60px 50px 60px]{hist-9b07b060359435635ff2bf4cd34f834a.pdf} \\ 
  num\_comments & The total number of comments (discussion and code review). & 0.00 & 2.59 & 1.00 & 11.00 & \includegraphics[scale = 0.1, clip = true, trim= 50px 60px 50px 60px]{hist-9db5e2b390de0d64d26c14798cb579ef.pdf} \\ 
  num\_participants & Number of participants in the pull request discussion & 0.00 & 1.27 & 1.00 & 4.00 & \includegraphics[scale = 0.1, clip = true, trim= 50px 60px 50px 60px]{hist-7d419bb69f175ea7015a9bdc71172f38.pdf} \\ 
  sloc & Executable lines of code at pull request merge time. & 469.00 & 56609.81 & 18493.50 & 295609.90 & \includegraphics[scale = 0.1, clip = true, trim= 50px 60px 50px 60px]{hist-6b5159d3060b4fdf8493d4c818f79949.pdf} \\ 
  team\_size & Number of active core team members during the last 3 months prior the pull request creation. & 1.00 & 21.11 & 7.00 & 93.00 & \includegraphics[scale = 0.1, clip = true, trim= 50px 60px 50px 60px]{hist-231fb4fabf4a3f0c551f2a97ae080508.pdf} \\ 
  perc\_external\_contribs & The ratio of commits from external members over core team members in the last 3 months prior to pull request creation. & 8.00 & 53.90 & 56.00 & 95.00 & \includegraphics[scale = 0.1, clip = true, trim= 50px 60px 50px 60px]{hist-a222f0a5c377ba129dd6c8f257062591.pdf} \\ 
  commits\_on\_files\_touched & Number of total commits on files touched by the pull request 3 months before the pull request creation time. & 0.00 & 51.49 & 4.00 & 208.00 & \includegraphics[scale = 0.1, clip = true, trim= 50px 60px 50px 60px]{hist-b735900ffcc37e7eda16dcd0c3497e6e.pdf} \\ 
  test\_lines\_per\_kloc & A proxy for the project's test coverage. & 0.00 & 1289.65 & 356.87 & 2093.26 & \includegraphics[scale = 0.1, clip = true, trim= 50px 60px 50px 60px]{hist-67ff3047089ba9ce0528884eab66e80a.pdf} \\ 
  prev\_pullreqs & Number of pull requests submitted by a specific developer, prior to the examined pull request. & 0.00 & 42.79 & 11.00 & 195.00 & \includegraphics[scale = 0.1, clip = true, trim= 50px 60px 50px 60px]{hist-a2f7f60851dfa13cfbe0227d1d233767.pdf} \\ 
  requester\_succ\_rate & The percentage of the developer's pull requests that have been merged up to the creation of the examined pull request. & 0.00 & 0.51 & 0.61 & 1.00 & \includegraphics[scale = 0.1, clip = true, trim= 50px 60px 50px 60px]{hist-9363017165c3ded62457750f1c67c1af.pdf} \\ 
   \hline
\end{tabular}
\caption{Selected features and descriptive statistics. Historgrams are in log scale.} 
\label{tab:features}
\end{table*}


\subsection{Merge detection}
\label{sec:mergedec}
To identify merged pull requests that are merged outside Github, we resorted to
the following heuristics, listed here in order of application:

\begin{description}

  \item[$H_1$] At least one of the commits associated with the pull request appears in the target project's master branch.

  \item[$H_2$] A commit closes the pull request (using the \texttt{fixes:}
    convention advocated by Github) and that commit appears in the project's
    master branch.  This means that the pull request commits were squashed onto
    one commit and this commit was merged.

  \item[$H_3$] One of the last 3 (in order of appearance) discussion comments
    contain a commit unique identifier, this commit appears in the project's
    master branch and the corresponding comment can be matched by the following
    regular expression:

    \begin{small}
    \texttt{(?:merg|appl|pull|push|integrat)(?:ing|i?ed)}
    \end{small}

  \item[$H_4$] The latest comment prior to closing the pull request matches the
    regular expression above.

\end{description}

If none of the above heuristics identifies a merge, the pull request is
identified as unmerged. In the dataset, 58\% of the pull requests are 
merged using Github's facilities while 18\% are identified as unmerged.
The remaining 24\% is identified as merged using the heuristics described
above. The contribution of each heuristic 

\subsection{Test case detection}

To evaluate testing related factors, the process exploits the fact that Java,
Ruby and Python projects are usually organized around the conventions imposed by
their build or library distribution systems (Maven, Gem and Pypi respectively).
Consequently, test cases are thus detected as follows:

\begin{description}

  \item[Ruby] Files under the \texttt{/test/} and \texttt{/spec/} directories
    are considered test files. Test cases are recognized by scanning through the
    test files lines for method name patterns as required by the \textsf{RUnit},
    \textsf{Rspec}, \textsf{Shoulda} and \textsf{Minitest} frameworks. The
    popular Cucumber behaviour driven development testing framework is excluded
    from the analysis due to inexistent naming conventions.

  \item[Python] Python project conventions do not specify specific directories
    for test files, so test file detection is based solely on whether the
    file name contains the word ``test'' as prefix or suffix. Test cases
    and asserts are discovered both in source code and also in API examples
    embedded in documentation comments.

  \item[Java] Following Maven conventions, files in directories under a
    \texttt{test/} branch of the file tree are considered test files.
    \textsf{junit4} test cases are recognized using the \texttt{@Test} tag. For
    \textsf{junit3}, methods starting with test are considered as test methods.
    Asserts are counted by searching through the source code lines for
    \texttt{assert} statements.

  \item[Scala] As Maven underlies Scala's default build system (\textsf{sbt}),
    the same conventions as in the Java case can be used to discover test files.
    In addition to \textsf{junit}, the process can discover test cases and
    assert statements as defined by the \textsf{scalatest} and
    \textsf{specs2} testing frameworks.

\end{description}

\subsection{Counting lines, files and file types}

Counting lines of code is an unsolved problem in software engineering research,
as researchers have yet to agree what constitutes a line of code in a source
code file. In the \pullreqs dataset, a line of code is an executable statement,
excluding blank lines and comments. To measure lines, we developed custom
comment strippers for all the programming languages the dataset supports, as
we could not find any tool that can count lines reliably (block comments in Ruby and Python were a particular problem).
Moreover, we delegated the identification of file types to a Ruby library
called Linguist (the same that Github uses), which supports more than 250 file types.
%~\footnote{The full list is here: \texttt{https://github.com/github/\-linguist/\-blob\-/master\-/lib\-/linguist/languages.yml}}

\section{Tools}

The \pullreqs dataset is accompanied by an extensive analysis toolkit written in
the R statistics language. The framework allows researchers to load selections
of projects in R, provides basic statistics in tabular (e.g.
Table~\ref{tab:features}) and graphical form (e.g. Figure~\ref{fig:features}),
and exposes a command line base interface that tools can extend. Moreover,
it implements a modular, multi-step data mining framework that researchers
can easily re-use in similar machine learning experiments. The data mining
parts include all the necessary steps for successfully executing a data mining
experiment, such as distribution plotting, cross correlation among features,
pluggable model and algorithm definitions, $n$-fold cross-validation and
plotting of important model variables such as Area Under Curve ({\sc auc}) and
accuracy across cross-validation runs. To cope with the \pullreqs dataset
size, several parts of the tooling employ parallelism and can thus exploit
multi-core machines.

%\section{Limitations}
%\label{sec:challenges}
%
%\paragraph*{Projects are incomplete} As alleged in Section~\ref{sec:expdata},
%
%
%\paragraph*{Merge detection is incomplete} 
%
%\paragraph*{Test detection is approximate}

\section{Related Work}
\label{sec:rel}

The \pullreqs dataset is similar to the code review datasets published by
Hamasaki et al.~\cite{Hamas13} and Mukadam et al.~\cite{Mukad13}.  It improves
upon those two datasets by covering a wider array of projects and languages
while also offering more precise related data, such as file counts and test
cases. To the best of our knowledge, this is the only publicly available
dataset covering pull-based development and certain aspects of distributed
software development in general.

\section{Conclusion}

We presented the \pullreqs dataset, a curated dataset of almost 800 projects,
along with a statistical analysis tool set for its analysis. New research directions whose analysis can be facilitated
by this dataset can include code review activities, recommender systems for pull
request handling and triaging and investigation of project process openness. The
dataset itself can be extended to support more programming languages and more
projects. Pull requests are especially welcome!

All source code and data is available on the Github repository
\href{https://github.com/gousiosg/pullreqs}{gousiosg/pullreqs}. Data on the
\href{http://ghtorrent.org}{ghtorrent.org} web site permit full replication
of the construction process or the expansion of this dataset.

\bibliographystyle{abbrv}
\balance
%\begin{small}

  \bibliography{dataset}
%\end{small}

\end{document}
